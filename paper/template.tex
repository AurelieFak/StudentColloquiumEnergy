\documentclass[journal]{vgtc}                % final (journal style)
%\documentclass[review,journal]{vgtc}         % review (journal style)
%\documentclass[widereview]{vgtc}             % wide-spaced review
%\documentclass[preprint,journal]{vgtc}       % preprint (journal style)
%\documentclass[electronic,journal]{vgtc}     % electronic version, journal

\let\ifpdf\relax

%% Uncomment one of the lines above depending on where your paper is
%% in the conference process. ``review'' and ``widereview'' are for review
%% submission, ``preprint'' is for pre-publication, and the final version
%% doesn't use a specific qualifier. Further, ``electronic'' includes
%% hyperreferences for more convenient online viewing.

%% Please use one of the ``review'' options in combination with the
%% assigned online id (see below) ONLY if your paper uses a double blind
%% review process. Some conferences, like IEEE Vis and InfoVis, have NOT
%% in the past.

%% Please note that the use of figures is not permitted on the first page
%% of the journal version.  Figures should begin on the second page and be
%% in CMYK or Grey scale format, otherwise, colour shifting may occur
%% during the printing process.  Papers submitted with figures on the
%% first page will be refused.

%% These three lines bring in essential packages: ``mathptmx'' for Type 1
%% typefaces, ``graphicx'' for inclusion of EPS figures. and ``times''
%% for proper handling of the times font family.

\usepackage{mathptmx}
\usepackage{graphicx}
\usepackage{times}

%% We encourage the use of mathptmx for consistent usage of times font
%% throughout the proceedings. However, if you encounter conflicts
%% with other math-related packages, you may want to disable it.

%% If you are submitting a paper to a conference for review with a double
%% blind reviewing process, please replace the value ``0'' below with your
%% OnlineID. Otherwise, you may safely leave it at ``0''.
\onlineid{0}

%% declare the category of your paper, only shown in review mode
\vgtccategory{Research}

%% allow for this line if you want the electronic option to work properly
\vgtcinsertpkg

%% In preprint mode you may define your own headline.
%\preprinttext{To appear in an IEEE VGTC sponsored conference.}

%% Paper title.

\title{The impact of user interface design of eco-feedback systems on consumer behavior}

%% This is how authors are specified in the journal style

%% indicate IEEE Member or Student Member in form indicated below
\author{Aur\'{e}lie Fakambie and Wouter Menninga}
\authorfooter{
%% insert punctuation at end of each item
\item
  Roy G. Biv is with Starbucks Research, E-mail: roy.g.biv@aol.com.
\item
  Ed Grimley is with Grimley Widgets, Inc., E-mail: ed.grimley@aol.com.
}

%% A teaser figure is NOT to be included.

%other entries to be set up for journal
\shortauthortitle{Biv \MakeLowercase{\textit{et al.}}: Global Illumination for Fun and Profit}
%\shortauthortitle{Firstauthor \MakeLowercase{\textit{et al.}}: Paper Title}

%% Abstract section.
\abstract{Saving energy in buildings has become and remains a major issue for the planet. The last decade, systems have been developed to provide consumers with information about their energy consumption. 
Research has shown that the type of information displayed and the techniques used to present it have an impact on the user energy saving. This raises the question about how to display the information to the consumer in a comprehensive, attractive and non-intrusive way. \\

In this paper we compare and discuss the various methods of visualizing energy usage for consumers. Some of the design components of user interfaces such as historical comparisons and presentation of costs are more likely to aid in providing the consumer with an understanding of his energy usage and changing his behavior.
We will extract the most effective methods from research and surveys. \\

The comparison of the different methods is based on the reduction of energy usage of consumers using such eco-feedback systems and if consumers keep using the eco-feedback systems for longer periods of time.

We expect to find the most effective methods to visualize energy consumption data for future eco-feedback systems.
} % end of abstract

%% Keywords that describe your work. Will show as 'Index Terms' in journal
%% please capitalize first letter and insert punctuation after last keyword
\keywords{Eco-Feedback, interface design, energy consumption, consumption feedback systems, energy feedback}

%% ACM Computing Classification System (CCS). 
%% See <http://www.acm.org/class/1998/> for details.
%% The ``\CCScat'' command takes four arguments.

\CCScatlist{ % not used in journal version
  \CCScat{K.6.1}{Management of Computing and Information Systems}%
{Project and People Management}{Life Cycle};
  \CCScat{K.7.m}{The Computing Profession}{Miscellaneous}{Ethics}
}

%% Copyright space is enabled by default as required by guidelines.
%% It is disabled by the 'review' option or via the following command:
% \nocopyrightspace

%%%%%%%%%%%%%%%%%%%%%%%%%%%%%%%%%%%%%%%%%%%%%%%%%%%%%%%%%%%%%%%%
%%%%%%%%%%%%%%%%%%%%%% START OF THE PAPER %%%%%%%%%%%%%%%%%%%%%%
%%%%%%%%%%%%%%%%%%%%%%%%%%%%%%%%%%%%%%%%%%%%%%%%%%%%%%%%%%%%%%%%%

\begin{document}

%% The ``\maketitle'' command must be the first command after the
%% ``\begin{document}'' command. It prepares and prints the title block.

%% the only exception to this rule is the \firstsection command
\firstsection{Introduction}

\maketitle

%% \section{Introduction} %for journal use above \firstsection{..} instead

% problem of energy usage
Reducing energy usage in buildings still remains a major challenge.

One method of reducing energy consumption is by increasing the awareness of consumers about their energy consumption using eco-feedback systems. These are systems with integrated sensors that provide the consumers in the building with information about their energy usage. The goal is that this leads to more energy efficient behavior by the consumers in the building. 

However, research has shown that the type of information displayed and the technique used to present it have an impact on the user behavior. This means that the design of the user interface is a key factor in changing the users energy consumption behavior.

Our goal is to investigate the different ways to display to the users their electricity usage.
The main UI components of eco-feedback systems are : historical comparison, presentation of costs, incentive, reward and commitment. From those components we want to extract the most effective ones : by effective we mean the ones which are more likely to help users save energy. Based on previous surveys we are going to compare percentage in the reduction of electricity usage according to the use of the different components. We are also going to compare different already existing UI. As criteria for the most effective ones studies have demonstrated that the information provided must be intuitive, clear and simple and the UI attractive and not too intrusive(e.g. not too many notifications) so that the users keep using it and is integrated in their everyday life.

This raises the question of what the most effective methods to visualize energy consumption data for future eco-feedback systems are. 

% to find the most effective methods to visualize energy consumption data for future eco-feedback systems.

% 'integrating sensors and systems to create eco feedback systems' which provide people in building with information about energy consumption behavior 
% many studies have shown that feedback works effectively...[ref]
%It has been shown in many studies that feedback systems have effect in reducing energy consumption
% design of user interface is a key factor to have impact on energy behavior
\section{User Interface components}
Blablabla

\section{The Surveys}
Several studies researching the effectiveness of consumer feedback on electricity consumption have been done before.
This section will discuss the results of some of these studies. \\

In a study from R.K. Jain et al\cite{jain2012assessing}, a prototype eco-feedback system was built, with five key design components:
\begin{itemize}
\item \textit{Historical comparison} - ability to view historical electricity consumption in three modes (24h, to date and last week)
\item \textit{Normative comparison} - ability to view the average electricity consumption of friends
\item \textit{Rewards and penalization} - ability to get points or lose points based on consumption behavior
\item \textit{Incentives} - ability to redeem points for prizes
\item \textit{Disaggregation} - ability to find out the consumption of specific devices
\end{itemize}

The prototype was designed to allow users to go to any of the key design components with a single click from the main view.

The system gathered and stored data on logins and use of the system in a database for later analysis.\\

Participants were divided into three groups: one group had access to room-level electricity usage data and consumption information for participants in their peer network added to the historical comparison graphs. 
The second group only had access to the room-level electricity usage data.
The third group was a control group without access to the eco-feedback system.

The researchers formulated and tested three hypothesis, namely:
\begin{enumerate}
\item Users who reduced their energy usage relative to the control group, will have visited the eco-feedback system more often than users who increased or maintained their energy usage.
\item Users that use: historical comparison, normative comparison, incentives/rewards or disaggregation will login more than users that do not use this feature.
\item The sign of the number of reward points a users view on their first login correlates with the number of times a user will log into the eco-feedback system.
\end{enumerate}

In Table~\ref{hypo1}, the results of performing an analysis of the login data can be seen. The data in this table confirms the hypothesis that users who decreases consumption logged in more often (almost twice as often) than users with an increased consumption.

\begin{table}
%% Table captions on top in journal version
  \caption{The results for the first hypothesis}
  \label{hypo1}
  \scriptsize
  \begin{center}
    \begin{tabular}{cccc}
    \multicolumn{1}{p{1cm}}{\centering } &
       \multicolumn{1}{p{2.5cm}}{\centering Users who reduced consumption} &
       \multicolumn{1}{p{2.5cm}}{\centering Users who increased consumption} &
       \multicolumn{1}{p{1cm}}{\centering p-value} \\
    \hline
      Mean user logins &  5.13 & 2.60 & .028\\

    \end{tabular}
  \end{center}
\end{table}

\begin{table}
  \caption{}
  \label{vis_accept}
  \scriptsize
  \begin{center}
    \begin{tabular}{lccc}
    \multicolumn{1}{p{2.5cm}}{\centering Mean user logins by component used} &
       \multicolumn{1}{p{2cm}}{\centering Users who used feature} &
       \multicolumn{1}{p{2cm}}{\centering Users who did not use feature} &
       \multicolumn{1}{p{1cm}}{\centering p-value} \\
    \hline
      Historical comparison &  4.61 & 1.67 & .0009\\
      Normative comparison &  5.00 & 2.40 & .12\\
      Incentives/rewards &  4.49 & 1.25 & .0001\\
      Disaggregation &  4.60 & 4.00 & .54\\

    \end{tabular}
  \end{center}
\end{table}

\begin{table}
  \caption{}
  \label{vis_accept}
  \scriptsize
  \begin{center}
    \begin{tabular}{lccc}
    \multicolumn{1}{p{2cm}}{\centering } &
       \multicolumn{1}{p{2cm}}{\centering Users that viewed positive points} &
       \multicolumn{1}{p{2cm}}{\centering  Users that viewed negative points} &
       \multicolumn{1}{p{1cm}}{\centering p-value} \\
    \hline
      Mean user logins &  4.79 & 2.10 & .0059\\

    \end{tabular}
  \end{center}
\end{table}

\section{Conclusion}


%% if specified like this the section will be ommitted in review mode
\acknowledgements{
The authors wish to thank A, B, C. This work was supported in part by
a grant from XYZ.}

\bibliographystyle{abbrv}
%%use following if all content of bibtex file should be shown
%\nocite{*}
\bibliography{template}
\end{document}
